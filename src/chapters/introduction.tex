Time series forecasting---predicting future values of variables within a domain---is a largely unsolved problem in complex or chaotic domains such as weather (e.g. humidity, temperature or wind speed) and economics (e.g. currency exchange rates or stock prices). In this thesis, we will attempt to solve the problem by using LSTM-based RNNs, something that has not been done before.

\section{Background and motivation}
Past attempts have approached the problem with algorithms that are not designed specifically to handle sequential datasets and instead assume that the data points are IID (Independent and Identically Distributed). Time series sequences have a temporal component dictating their intradependence and distribution within the dataset, where each element in the sequence is correlated to previous (with respect to time) elements through a plethora of factors.

The problem should be examined with algorithms, designed specifically for time series sequences, that take into account the temporal aspect of the datasets.

\section{Current research}
In the field of machine learning, HMMs (Hidden Markov Models), DBNs (Dynamic Bayesian Networks), TDNNs (Tapped-Delay Neural Networks) and RNNs (Recurrent Neural Networks) are commonly used to handle sequential datasets.

RNNs (Recurrent Neural Networks) have historically had problems with vanishing (or exploding) gradients, but this problem was solved by Hochreiter and Schmidhuber when they introduced their LSTM (Long Short-Term Memory) variant of RNNs. More recently, deep LSTM-based RNNs have successfully been applied to predict future time series sequences from historical, sequential datasets; such as predicting the weather for the next 24 hours, or predicting the next sequence of words in natural language.

\section{Problem statement}
Financial time series are used ubiquitously in algorithmic trading. In algorithmic trading, it is imperative that accurate predictions are made about numerous variables (such as volatility and returns) in order to time market entry and exit.

Since deep LSTM-based RNNs have only been applied within the algorithmic trading domain to a minimal extent, and since they have shown great success in solving similar problems in other domains, it raises the question whether the technique can be used to predict a future sequence of financial variables that can be used to time both entry and exit positions within a certain time horizon.

Presuming that correlations exist along the temporal dimension of the dataset, the problem is reduced to a matter of finding an appropriate set of \textit{featuers}, that enhance the correlations, on which to train the LSTM-based RNN.
