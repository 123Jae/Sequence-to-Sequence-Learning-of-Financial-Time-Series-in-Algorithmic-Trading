\end{multicols}
\chapter{Discussion}
\begin{multicols}{2}

\noindent Looking at the graphs (see Appendix A, B, C and D), already is it
evident that the application of our \textsc{lstm}-based \textsc{rnn} using
sequence-to-sequence learning is vastly superior to the traditional \textsc{rnn}
model.  This is also confirmed by looking at the evaluation measurements
(Table~\ref{tbl:results}) performed on the predictions: The \textsc{lstm}
outperforms the \textsc{rnn} on most evaluation measures, proving to be more
much more apt at modeling market behavior compared to the traditional
\textsc{rnn}.

\section{Financial Market Prediction}

Reaching back to the original research question; using the set of features
selected for the experiments setup in this thesis, the \textsc{lstm}-based
\textsc{rnn} was unable to model market behavior to such a degree that a
definitive call could be made as to whether the model generalized for the market
regime encountered during training or not.

Many attempts have been made at predicting financial market behavior by modeling
price returns, with some displaying a certain level of success.  In any case, a
larger data set could potentially provide more information, allowing the
\textsc{lstm} generalize better for market behavior.  That, however, would also
require a much larger model (i.e. a deeper model configuration with many more
cells in each layer) which, in turn, would place further demands on
computational capacity and time, which, in the end, put a limit on our ability
to pursue the problem further.  The more complex the model, and the larger the
dataset, the worse the \textsc{rnn} performs in comparison to the \textsc{lstm}.

\section{Visualizations}

Although the predictions are visualized \textit{(see Appendix A, B, C and D)} in
the form of graphs, there are interesting things to note.  Below, we reason
about the behavior of the models' ability in predicting market behavior.

The vanishing gradient problem becomes somewhat evident in the graphs (see
Appendix A, B, C and D).  The \textsc{rnn} is unable to produce much more than a
simple recurring pattern.

\subsubsection{Appendix A}

The predictions produced by the \textsc{lstm} (solid line) are intuitively
accurate and could possibly have been used for trading.  The large fall is not
predicted by the model, but the predicted price movements still follow the
market trend well up to that point.

Looking at the \textsc{rnn} (dashed line) predictions, there is no evident
generalization of market behavior.

\subsubsection{Appendix B}

Again, we see the general market trend predicted to an extent by the
\textsc{lstm}, giving an interesting result.

Looking at the \textsc{rnn} predictions again, we find no evidence of the market
behavior being modeled to any useful extent.

\subsubsection{Appendix C}

Although not giving accurate predictions, the \textsc{lstm} seems to have
modeled the \textit{double top chart pattern}: After two tops and a price fall
breaking the bottom of the trend channel; the model predicts a continuation of
the price drop, as would likely be expected by a human trader performing
technical trading analysis on the chart.

The \textsc{rnn} produces no useful predictions.

\subsubsection{Appendix D}

Here, we see the \textsc{lstm} model the general market trend over the
prediction period and onwards, but missing an important opportunity for
profitable returns; the large price movement during the prediction period.  The
\textsc{lstm} did not model the market behavior well enough.

The \textsc{rnn} produces no useful predictions.
