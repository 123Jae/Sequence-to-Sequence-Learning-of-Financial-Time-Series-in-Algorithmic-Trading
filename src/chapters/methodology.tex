\end{multicols}
\chapter{Methodology}
\begin{multicols}{2}

\noindent Up to this point, the theory behind algorithmic trading has been
explained thoroughly. In this chapter, the methods and approaches used to carry
out the experiments will be detailed.

\section{Approach}

There are, of course, different ways to approach the thesis problem and answer
the research question.  The approaches we picked for setting up the experiments
is not better or worse than other approaches in any definite or absolute manner.
The research question relies heavily on real-numbered evaluation measures,
narrowing down the number of feasible approaches to the quantitative kind.  The
candidate approaches and the rationale behind our final picks are laid out
below.

\subsection{Candidate Approaches}

We discussed several different approaches during the initial stages of our
thesis research.  Eventually, we came up with several candidate approaches, some
more relevant than others.  Many were discarded quickly, while others were
discussed thoroughly, with the more relevant ones detailed below.

\subsubsection{Autoencoder Feature Extraction}

Applying an autoencoder to the raw input data would produce an interesting
``compressed'' view of the data (bounded by entropy), potentially allowing us to
keep only relevant data.  This is mentioned in the Future Work chapter as it is
still a very interesting approach to the problem.  We decided against it due to
constraints on time and computational resources.

\subsubsection{Implementing a Trading Strategy}

The ultimate measure of performance of predictions, building a trading strategy
would give definite answers to the question regarding predicted market behavior.
Again limited by time, this was decided against.

\subsubsection{Statistical Evaluation Measures}

There are many statistical evaluation measures to choose from.  For the sake of
the thesis problem---comparing the \textsc{lstm} against the traditional
\textsc{rnn}---we found \textsc{mape} and \textsc{rmse} to be relevant.  We
decided to implement this approach as it provides real-numbered evaluation
measures, giving reproducible and testable results.

\subsubsection{Visualized Graphs}

Although not adequate for answering the research question on their own, graphs
provide an intuitive sense for the problem and our models' performance.  For the
skilled technical trader, this intuitive approach can be as useful as
real-valued measures. This approach was also implemented.

\subsection{Selected Approaches}

There are a couple of reasons for our particular choice of approach in
investigating the research question; first off, it is important that
real-numbered measures are presented as a way of displaying definite and
reproducible evaluation of the models' performance.  Secondly, although trading
algorithms rely entirely on \textit{data}, visualizing the results can give an
intuitive sense of the models' ability to predict market behavior: Although
these graphs do not answer any questions definitively, they serve the purpose of
reinforcing conclusions drawn from evaluation measures.

\subsection{Discarded Approaches}

There are many other methods for evaluating the performance with the most
important aspect of the predictions being whether they can be used for
successful trading on the Foreign Exchange Market.  For example, a buy-and-hold
decision based on predictions could be simulated, on which returns could be
computed, giving a trading performance estimate of the predictions.
Unforunately, there was not enough time to perform such experiments.

\section{Technology}

A plethora of techonology exists to aid in performing our particular thesis
experiments and although there are many viable alternatives, a choice had to be
made.  The rationale behind most choices made in regard to technology was
popularity (i.e. avoiding esoteric or unknown alternatives), availability,
performance and relevance to the scientific community, thus facilitating
reproducibility of our experiments and results to a greater extent.

\subsection{Docker}

A lesson learned; the same technology sometimes works different on different
computers.  Thankfully, Docker enabled us to reproduce and deploy an identical
environment on different machines, making collaboration much easier. Take note!

Although not required to perform the experiments, Docker made it so much easier
to collaborate using different hardware configurations, allowing everyone
involved to reproduce the setup on their machine.

\subsection{Python}

We used Python 3 (specifically version 3.5.2+) as the programming language of
choice as it is ubiquitously used by the scientific computing community and
academics all over the world.  Documentation is rich and there is a vast amount
of tools and third-party libraries available.

\subsection{Keras and TensorFlow}

Keras---a high-level deep learning library \citep{chollet2015}---was used to
implemented the machine learning algorithm models.  It is ideal as it provides
simple setup and execution of machine learning algorithms, making it almost
trivial to perform practical research within the field.

A TensorFlow \citep{tensorflow2015-whitepaper} backend was used in conjunction
with Keras, enabling us to easily perform computations on the computer's GPU and
increasing computational performance many times over.

Much happened during the time of writing the thesis in regard to the development
of both Keras and TensorFlow.  Despite both libraries providing rich
documentation, they both changed to such a degree during the writing of the
thesis that we had to rework our source code many times over.  During the first
weeks of writing the thesis, TensorFlow was not available on PyPI\footnote{PyPI
  is the Python Package Index, a repository of software for the Python
  programming language that can be installed via the \texttt{pip} command.} and
so we had to compile it manually.  As of finishing this thesis, both Keras and
TensorFlow can be installed easily via the Python package manager.

\subsection{Other Third-party Libraries}

For computational efficiency, pandas and numpy were our go-to solutions, providing
native code computational efficiency in Python.  Graphs and visualizations were
done with matplotlib. Keras uses h5py for saving and loading models in the
\textsc{hdf5} file format.

\section{Data}

Although there are many different financial assets and financial markets with
various degreens of liquidity etc, we needed freely available high-frequency
data to perform intraday trading experiments.  This fact limited our choices of
data sources, so many were ruled out early in the process.

\subsection{Selection}

The requirement on our part of high-frequency data to carry out the experiments
led us to the \textsc{forex} market, for which freely available market data can
be acquired.  We obtained the data from Dukascopy\footnote{Website:
  https://www.dukascopy.com/} by opening a trading account and downloading bulk,
high-frequency data.  The currency exchange between the Euro (€) and the
U.S. dollar (\$) was our choice of data for model training as it constitutes a
highly liquid financial market.

Beyond the selection of a data source, we had to limit the amount of data to put
through the neural model, being limited by computational resources.  We
initially decided to build a model for prediction of market behavior by looking
at the past three months of financial data (November through January), but later
decided on using the two weeks of trading data prior to the trading day to
perform predictions on.

The acquired data contained tick data, i.e.\ entries for each \textit{pip}
change, thus providing us with asks and bids, market volume and timestamps.
We did not have access to the \textit{market depth}---the number of bids and
asks at each price level.

\subsection{Preprocessing}

During preprocessing, irrelevant parts are removed, missing values are filled
in\footnote{Depending on the problem being solved, missing values can be
  ignored, interpolated or forward-filled.  In our case, forward-filling was the
  only viable alternative; using previous values and copying them forward in
  time to fill the holes in order to avoid look-ahead bias.}, normalization and
regularization is done and so on, essentially compiling the data to a set useful
for computing features on and using for model training.

All parameters relevant to training of our network model---considering our
technical analysis approach---were already part of the acquired high-frequency
data.

The pip data was aggregated into one-minute candlesticks, containing
\textsc{ohlc} prices, with missing candlesticks forwared-filled from the
previously existing candlestick in order to avoid look-ahead bias.

The magnitude of price changes on the \textsc{forex} market are small---which
poses a problem for the neural network model as explained in the theory
chapter---but it could not be normalized over an entire data set as that would
violate the rule of not looking into the future, since, in a live trading
scenario, the model would only have access to already-seen prices, which would
make it impossible to normalize the price with future prices in mind.  Instead,
this was solved during feature extraction through careful choice of features.

\section{Evaluation}

The models were evaluated in several ways.  Partly, the models' predictions were
visualized to provide an intuitive overview of their performance.  To provide a
real-numbered estimate of their performance, measurements were computed on their
predictions.

\subsection{Visualization}

Although merely giving an intuitive sense of the models' performance,
visualizing the results still provides an important insight into to the
performance of the models.  The graphs are interesting to look at in comparison
to each other; it is evident that the \textsc{lstm} model is superior to the
traditional \textsc{rnn}, displaying a higher ability in predicting market
behavior \textit{(see Appendix A, B, C and D)}.

Furthermore, these graphs were used to confirm the integrity of the
models---that they did what they were supposed to do.

\subsection{Evaluation Measures}

A more important aspect of the model evaluation process is the quantification of
the models' performance, using real-valued measurements.  Using the mean
absolute percentage error (\textsc{mape}) and root-mean-square error
(\textsc{rmse}), the models were evaluated and compared against each other.  The
models were evaluated at several points in time to avoid serendipitous results.

\subsubsection{Mean Absolute Percentage Error (\textsc{mape})}

The Mean Absolute Percentage Error is a measure of prediction accuracy of a
forecasting method.  It is used to compute the accuracy of a trend estimation as
a percentage.

\begin{Figure}
  \begin{align*}
    & \qquad\qquad\qquad M=\frac{100}{n}\sum\limits_{t=1}^n\ \abs{\frac{A_t-F_t}{A_t}} \\
    & \text{where $A_t$ is the actual value and $F_t$ is the forecast value}
  \end{align*}
  \captionof{eqn}{The mean absolute percentage error formula.}
\end{Figure}

\subsubsection{Root-Mean-Square Error (\textsc{rmse})}

The Root-Mean-Square Error is a measure of difference between actual values and
values predicted by a model.  It represents a standard deviation of the
differences between actual predicted values.

\begin{Figure}
  \begin{align*}
    & \qquad\qquad\qquad {RMSE}(\hat{\theta}) = \sqrt{{MSE}(\hat{\theta})} \\
    & \text{where ${MSE}$ is the mean square error of the predicted} \\
    & \text{values.}
  \end{align*}
  \captionof{eqn}{The root-mean-square error formula.}
\end{Figure}