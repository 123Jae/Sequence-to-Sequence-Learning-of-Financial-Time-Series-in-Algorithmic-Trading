Studien utfördes för att kunna analysera svårigheterna att förutspå framtida
värden i en kaotisk marknad med dem senaste teknikerna för neurala nätverk. Då
problemet handlar om att tolka stora mängder data gjorde vi en kvantitativ
ansats.

\section{Forskningstrategi}
Den kvantitativa ansatsen gjordes för att tolka all data över valutahandel som
det neurala nätverket fick träna på. Vi studerade dess predicerade resultat för
en valutas framtida riktning. Riktningen på en valutas värde kunde vara antingen
uppgång, nedgång eller stillestånd.

\section{Valda Tekniker}
Det neurala nätverket implementerades i Python 2.7 med biblioteket TensorFlow
\citep{tensorflow2015-whitepaper}. Det neurala nätverket var ett \textsc{rnn} 
med \textsc{lstm}. Båda teknikerna \textsc{(rnn \& lstm)} finns implementerade i 
detta bibliotek. En avgränsning som behövde göras var just att använda TensorFlow.
Det hade varit intressant att
göra jämförelser mellan TensorFlow och andra stora bibliotek med
implementeringar av neurala nätverk så som Theano, Keras, Lasagne, Blocks,
SciKit m.m.. Att göra en egen implementering av ett neuralt nätverk för djupare
konfigureringar fick även det prioriteras bort. Specifika hyperparametrar för
TensorFlow ändrades under arbetets gång för att optimera nätverket. Olika
storlekar på tidsfönstret testades för att hitta ett optimalt sätt för nätverket
att förutspå valutan.

\section{Datainsamling}
Studien är avgränsad mot valutahandel då högfrekvens-data över aktier har en hög
efterfrågan. Det gör att den är svår, men framförallt dyr, att få tag på. För
att kunna genomföra studien behövdes data med hög frekvens samt att den har
tillräckligt med parametrar utan att för den delen bli inkomplett. Datan som
användes hämtades från foreign exchange market \textsc{(forex)} och visar
högfrekvenshandel med valutor, där fanns parametrarna som behövdes för att träna
nätverket.

\section{Dataanalysmetod}
Resultaten som nätverket gav jämfördes sedan med korrekt data för samma
tidsperiod och en slutsats kunde lätt dras angående modellens
pricksäckerhet. Det gjordes även jämförelser mot andra
"state-of-the-art"-algoritmer implementerade i just TensorFlow, detta för att ha
möjlighet att utvärdera effektivititen av studiens nätverk.
